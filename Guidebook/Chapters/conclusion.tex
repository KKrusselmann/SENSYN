%%%%%%%%%%%%%%%%%%%%%%%%%%%%%%%%%%%%%%%%
% Chapter 6: Conclusion
%%%%%%%%%%%%%%%%%%%%%%%%%%%%%%%%%%%%%%%%

\label{conclusion}

Strict regulations surrounding personal and other sensitive data can seem at odds with the implementation of Open Science principles, creating hinderances for researchers, research support and other stakeholders. Synthetic data carries many promises for enhancing data sharing practices, open datasets and the implementation of FAIR principles, in particular for sensitive data. It allows for the creation of artificial datasets that mimic real-world data whilst protecting private and personal information. These datasets can be shared, registered, fit to other open science standards and be used without many restrictions. So far, synthetic data is mostly discussed amongst data scientists, who develop new measures of utility and privacy, new computations for syntheses processes and advance already existing approaches. However, synthetic data is not applied much in fields other than data science. With fast developments in artificial intelligence and data science, synthetic data promises to become increasingly useful in the future.  \\

We hope this guidebook provides a short introduction to the ideas and implementation of synthetic data, as well as practical tips on how synthetic data can be applied in everyday research to make it more open, findable, interoperable, accessible, and reusable.  


